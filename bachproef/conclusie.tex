%%=============================================================================
%% Conclusie
%%=============================================================================

\chapter{Conclusie}
\label{ch:conclusie}

% TODO: Trek een duidelijke conclusie, in de vorm van een antwoord op de
% onderzoeksvra(a)g(en). Wat was jouw bijdrage aan het onderzoeksdomein en
% hoe biedt dit meerwaarde aan het vakgebied/doelgroep? 
% Reflecteer kritisch over het resultaat. In Engelse teksten wordt deze sectie
% ``Discussion'' genoemd. Had je deze uitkomst verwacht? Zijn er zaken die nog
% niet duidelijk zijn?
% Heeft het onderzoek geleid tot nieuwe vragen die uitnodigen tot verder 
%onderzoek?

Uit het onderzoek kan men concluderen dat dit geen onderzoeksvragen zijn waar men een eenduidig antwoord kan op geven. Als men zonder bijkomende vragen hier zou op antwoorden, zou het antwoord Wordpress zijn. Het is het content management systeem waarvoor men over de minste technische kennis moet beschikken. Samen met de WooCommerce extensie is men dan ook perfect in staat om een e-commerce oplossing uit te bouwen. Natuurlijk is het niet zo makkelijk.

Voor de eerste onderzoeksvraag moet men rekening houden met twee zaken de beschikbare tijd en de toekomst van het project. Als men een kleine tot middelgrote website wenst op te bouwen in een korte tijd maakt men best gebruik van Wordpress. Als men over meer tijd beschikt kan men ervoor opteren om gebruik te maken van Joomla. Als men een grote website met een ruime tijdspanne kan men best gebruik maken. Zo kan men de leercurve dat verbonden is aan Drupal overbruggen en optimaal gebruik maken van alle opties. Voor de tweede onderzoeksvraag moet men ook weer rekening houden met de eerder vernoemde factoren. Als men op korte tijd een kleine tot middelgrote webshop uitbouwen kan men best gebruik maken van Wordpress met WooCommerce. Als er meer tijd is voor het uitwerken van de webshop kan men ervoor opteren om gebruik te maken van Joomla met VirtueMart. Als men een grote en complexe webshop moet uitwerken kan men best gebruik maken van Drupal met Drupal Commerce. Deze antwoorden zijn gebaseerd op de resultaten van de kritische meetpunten.

Afgaande uit de resultaten van dit onderzoek zou men Wordpress nog altijd tussen Joomla en Drupal plaatsen. Drupal is nog altijd de beste optie voor een complexe, meertallige webshop. Men zou niet meer volmondig ja kunnen zeggen als men vraagt of dat Drupal nog steeds de beste keuze is voor een e-commerce oplossing. Men zou dan beter de vraag stellen wat is het scenario en hoeveel tijd heeft men. 
 
In de toekomst kan dit onderzoek gebruikt worden als basis voor andere onderzoeken. Zo kan men controleren hoe deze meetpunten geëvolueerd zijn in de nieuwste versies van deze content management systemen. Men kan het huidige onderzoek ook uitbreiden door extra meetpunten toe te voegen of meerdere content management systemen en e-commerce uitbreidigen toe te voegen.