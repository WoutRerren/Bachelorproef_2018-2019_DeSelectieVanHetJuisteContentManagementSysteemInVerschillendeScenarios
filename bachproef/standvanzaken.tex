\chapter{\IfLanguageName{dutch}{Stand van zaken}{State of the art}}
\label{ch:stand-van-zaken}

% Tip: Begin elk hoofdstuk met een paragraaf inleiding die beschrijft hoe
% dit hoofdstuk past binnen het geheel van de bachelorproef. Geef in het
% bijzonder aan wat de link is met het vorige en volgende hoofdstuk.

% Pas na deze inleidende paragraaf komt de eerste sectiehoofding.
In dit hoofdstuk zal er een introductie volgen van enkele belangrijke termen die meermaals zullen terugkeren in het onderzoek. Hier zal er ook een korte geschiedenis meegegeven worden over de content management systemen en de e-commerce uitbreidingen dat gekozen werden voor het onderzoeken.
\section{E-Commerce}
\textit{E-commerce, vaak ook wel E-Business genoemd, is de verzameling van alle transacties die op een digitale manier gebeuren. Het bekendste voorbeeld hiervan is natuurlijk, het aankopen van een artikel op een online webshop. De term e-commerce dateert van de jaren '70, hier werd deze gebruikt voor het beschrijven van het systeem dat de banken gebruikt voor het onderling uitwisselen van geld. De laatste jaren is deze term razend populair geworden door de enorme stijging in het online winkelen. Ondersteunde activiteiten worden vaak ook als e-commerce gezien. Denk maar aan zaken zoals het afhandelen van online betalingen, leveren van de goederen en digitale marketing.} \autocite{MarketingTermen2018}
\subsection{WooCommerce}
WooCommmerce is een WordPress plugin dat je gewone website verandert in een e-commerce oplossing. Uit de stastieken van \textcite{Builtwith2018a} kunnen we het een en ander afleiden. In het totaal zijn er 2 906 789 websites waar dat er WooCommerce opdraait. In het totaal van de 1 miljoen populairste sites zijn er 54 123 websites die WooCommerce draaien. Dit resulteert dus in een percentage van 5.41\%.\footnote{Ze bepalen de populariteit op basis van het verkeer naar deze websites}\footnote{Deze getallen zullen wel nog hoger liggen, dit gaat enkel over de websites waar zij toegang tot hebben}Op de website van \textcite{Wordpress2018} kan men de download geschiedenis van verschillende plugins zien. Ten tijden van het schrijven van deze scriptie werd WooCommerce tijdens de afgelopen zeven dagen 323 799 keer gedownload. \footnote{December 2018} Door de grote populariteit van WooCommerce werd er gekozen om WooCommerce als e-commerce extensie te gebruiken voor Wordpress.
\subsection{Drupal Commerce}
Drupal Commerce is een Drupal module dat een Drupal website verandert in een e-commerce oplossing. Drupal Commerce wordt onderhouden door de Commerce Guys. Er is een contribute mogelijkheid waardoor men kan meehelpen aan de ontwikkeling van Drupal Commerce. Volgens de \textcite{Drupal2018a} website is de module al 1 013 898 keer gedownload. Uit de statistieken van \textcite{Builtwith2018b} zien we dat er een totaal van 5 799 websites live zijn met Drupal Commerce. Het interessantste aan Drupal Commerce is de Commerce Kickstart installation profile. Hierdoor installeert men drupal met een volledige geconfigureerd Drupal Commerce module. Dit verklaart de keuze van Drupal Commerce als e-commerce extensie voor Drupal.
\subsection{VirtueMart}
Door het toevoegen van de VirtueMart extensie aan een bestaande Joomla website verandert men een  website in een oogopslag in een e-commerce oplossing. Uit de statstieken van \textcite{Builtwith2018c} kunnen we afleiden dat er 60 585 websites live zijn met de VirtuMart extensie. De geschiedenis van deze extensie is gelijklopend met die van Joomla. Vandaar de keuze voor VirtueMart als e-commmerce extensie voor Joomla.

\section{Content Management Systemen}
\textit{Content Management Systeem, of afgekort CMS, is een systeem dat achter de website zit. Dit systeem stelt jou in staat om makkelijk jouw website te beheren. Deze systemenen zijn vaak voorzien van een eenvoudige interface waar je zonder enige kennis van Hypertext Markup Language (HTML), Cascading Style Sheets (CSS) en programmeertalen waarmee je makkelijk een website kan opbouwen. De beheerder kan hierdoor eenvoudig content toevoegen en verwijderen. De toevoegingen gebeuren meestal aan de hand van een WYSIWYG-editor(what you see is what you get),zo een editor komt veel mensen bekend voor, gezien de vele gelijkenissen met het populaire tekstverwerking programma Word.} \autocite{forresult2014,wphandleiding2015}


Uit de statistieken van \textcite{Builtwith2018} blijkt dat op 54\% van alle websites op het internet een Content Management Systeem draait.\textit{ Van die 54\% draait 32.9\% op Wordpress wat dus resulteert in een marktaandeel van 59,6\%. Joomla draait op 3.0\% van de 54\% wat resulteert in een marktaandeel van 5.6\%. Drupal draait op 1.9\% van de 54\% wat resulteert in een marktaandeel van 3.6\%.} \autocite{Builtwith2018} Uit deze statistieken kunnen we ook makkelijk afleiden dat Wordpress,Joomla en Drupal de drie populairste Content Management Systemen zijn. Logischerwijs zijn dit ook de content management systemen die in dit onderzoek onderzocht worden. 


\subsection{Wordpress}
\textit{Wordpress is een open-source content management systeem, open-source wilt zeggen dat iedereen toegang heeft tot de broncode en hier aan kan bijdragen. Het avontuur van Wordpress begon in 2003 met Matt Mullenweg en Mike Little. Het startte toen als een fork van het populaire blogsysteem b2/cafelog. Toen er plots geen updates meer kwamen van het populaire blogsysteem lanceerde Matt het idee om een nieuw systeem te ontwikkelen op basis van b2/cafelog. Mike hapte onmiddellijk toe en zo startten ze samen het avontuur dat nu Wordpress heet. Wat startte als een blogsysteem is ondertussen uitgegroeid tot een volwaardig CMS met een heel ecosysteem aan plug-ins en themes}.\autocite{Postma2018}

\subsection{Drupal}
Net zoals Wordpress is Drupal ook een open-source content management systeem. \textit{In 2000 lanceert de Antwerpse universiteit student Dries Buytaert een kleine content framework. Het framework draaide op het interne netwerk dat hij en medestudenten opgezet hadden tussen hun kamers. Het framework diende voor hun onderling communicatie zoals de status van het netwerk dat ze opgezet hadden, waar ze de volgende dag zouden eten,etc. Nadat Dries afstudeerde, lanceerde hij het framework online zodat hij en zijn kameraden in contact konden blijven. Drop.org was geboren. Al snel trok het een heel andere publiek aan, er werd steeds gesproken over de nieuwste trends binnenin webtechnologieën, deze trends werden uiteindelijk methoden en kenmerken van de software waarop drop.org draaide. In janauri 2001 maakte hij de software achter drop.org open-source, nu kon iedereen vrij de software uitbreiden met hun nieuwste ideeën. Hij noemde de software Drupal.}\autocite{Drupal2018}

\subsection{Joomla}
Joomla is net zoals Wordpress en Drupal ook een open-source content management systeem. \textit{In 2001 brengt het Australische bedrijf Miro het open-source CMS Mambo uit. Mambo was vlak na de release vrij populair, maar na een meningsverschil tussen het Mambo core team en het besturingscomitée, nam het grootste deel van het mambo core team ontslag. Dit resulteerde in de entiteit Open Source Matters en een code fork van Mambo genaamd Joomla. In 2005 werd de eerste versie van Joomla gelanceerd, dit was voornamelijk rebranding en enkel updates in vergelijking met het originele Mambo. Ondertussen is Joomla uitgegroeid tot een internationaal erkend content management systeem, Mambo daarentegen wordt enkel nog vermeld om de geschiedenis van Joomla te beschrijven.}\autocite{Crowder2009} .