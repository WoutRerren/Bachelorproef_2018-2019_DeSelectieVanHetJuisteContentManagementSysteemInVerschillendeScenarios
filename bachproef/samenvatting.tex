%%=============================================================================
%% Samenvatting
%%=============================================================================

% TODO: De "abstract" of samenvatting is een kernachtige (~ 1 blz. voor een
% thesis) synthese van het document.
%
% Deze aspecten moeten zeker aan bod komen:
% - Context: waarom is dit werk belangrijk?
% - Nood: waarom moest dit onderzocht worden?
% - Taak: wat heb je precies gedaan?
% - Object: wat staat in dit document geschreven?
% - Resultaat: wat was het resultaat?
% - Conclusie: wat is/zijn de belangrijkste conclusie(s)?
% - Perspectief: blijven er nog vragen open die in de toekomst nog kunnen
%    onderzocht worden? Wat is een mogelijk vervolg voor jouw onderzoek?
%
% LET OP! Een samenvatting is GEEN voorwoord!

%%---------- Nederlandse samenvatting -----------------------------------------
%
% TODO: Als je je bachelorproef in het Engels schrijft, moet je eerst een
% Nederlandse samenvatting invoegen. Haal daarvoor onderstaande code uit
% commentaar.
% Wie zijn bachelorproef in het Nederlands schrijft, kan dit negeren, de inhoud
% wordt niet in het document ingevoegd.

\IfLanguageName{english}{%
\selectlanguage{dutch}
\chapter*{Samenvatting}


\selectlanguage{english}
}{}

%%---------- Samenvatting -----------------------------------------------------
% De samenvatting in de hoofdtaal van het document

\chapter*{\IfLanguageName{dutch}{Samenvatting}{Abstract}}

%\lipsum[1-4]
Vandaag de dag worden content management systemen meer en meer gebruikt. De voorbije jaren zijn deze systemen bezig met een forse opmars. Het content management systeem heeft de voorbije jaren dus enorm aan populariteit gewonnen. De reden hiervoor is dat door deze systemen mensen zonder enige technische kennis in staat zijn om een website te bouwen. In dit onderzoek zal men een vergelijking maken tussen de bekendste content mangement systemen, namelijk Wordpress, Joomla en Drupal.

Deze systemen zijn de voorbije jaren zodanig geëvolueerd dat men nu al in staat is om webshops uit te bouwen. Hierdoor komen er steeds meer en meer nieuwe e-commerce oplossingen bij. In dit onderzoek zal men een vergelijking maken tussen drie e-commerce uitbreiding van de content mangement systemen van het eerste deel van dit onderzoek. Deze uitbreidingen zijn WooCommerce, VirtueMart en Drupal Commerce.

Men zal een vergelijkende studie maken tussen de verschillende systemen en e-commerce uitbreidingen op basis van verscheidende meetpunten. Deze meetpunten zijn bepaald op basis van gelijkaardige onderzoeken en belangrijke factoren die een rol spelen tijdens het onderhouden, gebruiken en uitbouwen van een website.

Deze studie heeft als doel om mensen met een niet-technische achtergrond te helpen met hun keuze voor het uitwerken van hun website of e-commerce oplossing. Mensen met een technische achtergrond kunnen hier natuurlijk ook geholpen worden met hun keuze.